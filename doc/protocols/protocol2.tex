\documentclass[DIN, pagenumber=false, fontsize=11pt, parskip=half]{scrartcl}

\usepackage{ngerman}
%\selectlanguage{english}
\usepackage[utf8]{inputenc}
\usepackage[T1]{fontenc}
\usepackage{textcomp}
\usepackage{float}

% for matlab code
% bw = blackwhite - optimized for print, otherwise source is colored
%\usepackage[framed,numbered,bw]{mcode}

% for other code
\usepackage{listings}

\setlength{\parindent}{0em}
\usepackage{graphicx}

% set section in CM
\setkomafont{section}{\normalfont\bfseries\Large}

%===================================
\begin{document}           % typeset the header of the contribution


Decisions:\\
We will start with the gathering of data:\\
\begin{itemize}
\item weather data from online
\item sensor data (Humidity and Temperature)
\item need to order a light sensor
\item need to order a CO2 sensor
\end{itemize}

Actuators will be decided next time. For now most likely mail information about windows/small leds symbolizing the lights in the room.\\
TODO find out if our smart home plugs are compatible.\\

%
First, you mention a learning algorithm. What will this algorithm learn and why does the system need it? \\
Learning algorithm will be moved to additional features for now.\\

Second, what will the data processor do? I see a Plugwise processor. Do you have Plugwise plugs? \\
Going around too expensive hardware by using some senosors we already have.\\

Will the system store the data somewhere? \\
The data will be stored in a data folder. The sampling rate must be tested yet. 

Will the system use stored data? It seems that the PlannerService will send actions to the DataProcessor component. Is this intentional? \\
need to discuss this topic once we start compiling everything together.

Why not having a separate execution component? \\

What about connecting the PlannerService with the user interface? \\
The UI mus be able to communicate with the PI to get current samples.

\end{document}
